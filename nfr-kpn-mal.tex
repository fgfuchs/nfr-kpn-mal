\documentclass[11pt,a4paper,norsk]{article}
\usepackage{amsmath}
\usepackage{unicode-math}
\unimathsetup{math-style=ISO,bold-style=ISO, % ISO standards for italic & bold
	nabla=upright, partial=upright, % Operators should be upright
}

\usepackage{babel}

\usepackage[margin=2cm,top=1.25cm,bottom=2.5cm,footskip=0.65cm]{geometry}
\usepackage{ragged2e}
\RaggedRight

\usepackage{parskip}
\parskip=\baselineskip \advance\parskip by 0pt plus 2pt

\usepackage{fontspec}
\setmainfont{Times New Roman}
\setmathfont{XITS Math}
\setsansfont{Calibri}
\linespread{0.93} % MS Word uses 1.15 * 11 pt base line skip, while LaTeX uses 13.6 for 11 pt font. Need a line spread of 1.15 / (13.6 / 11) = 0.93 to match.

\date{}
\author{}

\usepackage{titling}
\pretitle{\Large\bfseries}
\posttitle{\par}
\preauthor{}
\postauthor{\par}
\predate{}
\postdate{\par\vskip -1.5em}
\setlength{\droptitle}{-60pt}

\usepackage{titlesec}
\titlespacing*{\part}{0pt}{0pt}{0pt}
\titlespacing*{\section}{0pt}{0pt}{-1\parskip}
\titlespacing*{\subsection}{0pt}{0pt}{-1\parskip}
\titlespacing*{\subsubsection}{0pt}{0pt}{-1\parskip}
\titlespacing*{\paragraph}{0pt}{0pt}{1em}
\titlespacing*{\subparagraph}{17pt}{0pt}{1em}
\titleformat{\part}{\normalfont\large\bfseries}{PART~\arabic{part}:}{1em}{\normalfont\large\bfseries}
\titleformat{\section}{\normalfont\normalsize\bfseries}{\thesection.}{1em}{\normalfont\normalsize\bfseries}
\titleformat{\subsection}{\normalfont\normalsize\bfseries}{\thesubsection.}{1em}{\normalfont\normalsize\bfseries}
\titleformat{\subsubsection}{\normalfont\normalsize\bfseries}{\thesubsubsection.}{1em}{\normalfont\normalsize\bfseries}

\usepackage{fancyhdr}
\usepackage{lastpage}
\fancypagestyle{plain}{%
\fancyhf{} % clear all header and footer fields
\fancyfoot[R]{\footnotesize side \thepage/\pageref*{LastPage}}
\fancyfoot[C]{\footnotesize Kompetanseprosjekt for næringslivet, mal for prosjektbeskrivelse 07.12.2010}
\renewcommand{\headrulewidth}{0pt}
\renewcommand{\footrulewidth}{0pt}}
\pagestyle{plain}

\usepackage{tabulary}
\renewcommand{\arraystretch}{1.1}
\setlength{\arrayrulewidth}{0.5pt}

\usepackage{xcolor}
\definecolor{tangored}{HTML}{CC0000}
\definecolor{tangogreen}{HTML}{4E9A06}
\definecolor{tangoblue}{HTML}{3465A4}
\usepackage{hyperref}
\hypersetup{
	colorlinks,
	linkcolor={tangored},
	citecolor={tangogreen},
	urlcolor={tangoblue}
}

\usepackage[strict, english=british]{csquotes}
\MakeOuterQuote{"}

\frenchspacing


\usepackage{mathtools}
\mathtoolsset{mathic}

\usepackage[capitalize]{cleveref} % Load last

\title{Project description}

\begin{document}

\maketitle

\part{Knowledge needs}

\section{Knowledge needs}
Beskriv de overordnede kunnskapsutfordringer og kompetansebehov som begrunner hvorfor prosjektet skal igangsettes.

\part{The knowledge-building project}

\section{Objectives}
Formuler et konkret og etterprøvbart hovedmål for kompetanseprosjektet. Formuler punktvis etterprøvbare delmål som leder frem til hovedmålet. Det må klart fremgå hvilke resultater som forventes oppnådd.

\section{Frontiers of knowledge and technology}
Gjør rede for status nasjonalt og internasjonalt (state of the art) for de relevante teknologier/forskningstemaer prosjektet retter seg mot.

Beskriv på hvilken måte prosjektet vil frembringe ny kunnskap av betydning for den faglige utviklingen på disse feltene.

\section{Research tasks and scientific methods}
Beskriv og avgrens problemstillingen i prosjektet. Beskriv de mest sentrale forskningsoppgavene i lys av de kompetansebehov prosjektet retter seg mot.

Beskriv hvilke metoder og teorier som tenkes brukt, og vis at de er godt egnet til å besvare problemstillingen, eller at det er gode muligheter for å utvikle den nødvendige metode og teori.

Beskriv hvordan stipendiater er knyttet til forskningsoppgavene i prosjektet.

Beskriv også publisering i vitenskapelige tidsskrifter med refereeordning.

\section{Organisation and project plan}
Beskriv hvilken rolle hver av forskningspartnerne (inklusive utenlandske partnere) har i gjennomføringen og hvilken kunnskap/kompetanse de bidrar med.

Beskriv hvilken rolle hver av bedriftspartnerne og eventuelle andre brukere vil ha i styringen og gjennomføringen av prosjektet, utover sitt finansieringsbidrag. Dersom bedriftene vil ha relevante FoU-aktiviteter ved siden av prosjektet, beskriv dette nærmere.

Begrunn samarbeidskonstellasjonen i prosjektet. Beskriv spesielt hvilke vurderinger som er gjort angående behov for internasjonalt samarbeid i prosjektet.

Bruk \cref{tab:project-plan} for å vise mål og leveranser med tilhørende kostnader for alle hovedaktivitetene i prosjektet, slik disse er angitt i søknadsskjemaets tabell "Hovedaktiviteter og milepæler i prosjektperioden". Summen av alle kostnadene skal stemme med totalkostnaden for prosjektet, ref søknadsskjemaet pkt. "Kostnadsplan". Angi også hvilken partner som er ansvarlig for aktiviteten, og hvilke andre som deltar.

\begin{table}
	\centering
	\begin{tabulary}{\textwidth}{|L|L|L|L|L|}
		\hline
		No. & Main activity, goals and deliverables & Total expenses & Resposible partner & Participating partners \\ \hline
		1   &                                       &                &                    &                        \\ \hline
		    &                                       &                &                    &                        \\ \hline
	\end{tabulary}
\caption{Project plan\label{tab:project-plan}}
\end{table}

\section{Key milestones}
Beskriv og tidfest de milepæler som anses avgjørende for gjennomføring av prosjektet og hvilke delmål disse er knyttet til. Dette vil typisk være beslutningspunkter som kan innebære viktige veivalg i prosjektgjennomføringen.

\section{Costs incurred by each research performing partner (NOK 1 000)}
Det skal gis en oversikt som viser hvordan prosjektkostnadene fordeler seg på hver av de utførende forskningspartnerne. Bruk \cref{tab:costs} til å vise prosjektkostnadene fordelt på personal- og indirekte kostnader, utstyr, andre kostnader og totalt per partner. Summen av alle kostnadene skal stemme med totalkostnaden for prosjektet, ref søknadsskjemaet pkt. "Kostnadsplan".

\begin{table}
	\centering
	\begin{tabulary}{\textwidth}{|L|L|L|L|L|}
		\hline
		Research performing partner & Payroll and indir. exp. & Equipment & Other op. expenses & Totals \\ \hline
		                            &                         &           &                    &        \\ \hline
		                            &                         &           &                    &        \\ \hline
	\end{tabulary}
	\caption{Costs\label{tab:costs}}
\end{table}

\section{Financial contribution by industrial partner or other user (NOK 1 000)}
Vis i \cref{tab:financial-contrib} den enkelte bedriftspartners kontantbidrag. Siste rad i tabellen skal vise det beløp som er søkt fra Forskningsrådet. Dette beløpet kan være maksimalt 4 ganger større enn det samlede kontantbidraget fra bedrifts-/brukerpartnerne. Summen av kontantbidragene og søkt beløp fra Forskningsrådet må stemme med totalsummen for prosjektet, ref. søknadsskjemaet pkt. "Finansieringsplan".

\begin{table}
	\centering
	\begin{tabulary}{\textwidth}{|L|L|}
		\hline
		Industrial partner or other user & Cash financing \\ \hline
		                                 &                \\ \hline
		                                 &                \\ \hline
		From Research Council            &                \\ \hline
	\end{tabulary}
	\caption{Financial contributions\label{tab:financial-contrib}}
\end{table}

\section{Other collaboration}
Dersom prosjektet innebærer samarbeid også med andre enn de partnere som er angitt i kostnads- og finasieringstabellene, skal dette beskrives her.

\part{Project impact}

\section{Importance for national knowledge base}
Beskriv på hvilken måte prosjektet vil bidra til den langsiktige kompetanseoppbyggingen nasjonalt på de aktuelle områdene, for eksempel ved utvikling av spisskompetanse, utvidelse av kompetansebasen, forskerutdanning eller for utvikling av relevante undervisningstilbud.

Redegjør for forskningspartnernes kvalifikasjoner og forskningskapasitet på det aktuelle området, samt deres nasjonale posisjon på feltet. Beskriv hvordan prosjektet passer inn i strategier og planer hos forskningspartnerne.

Beskriv hvorfor det er viktig at denne kompetansen bygges opp i Norge fremfor å anskaffe den fra utlandet.

\section{Relevance for Norwegian industry}
\begin{itemize}
	\item[a)] Beskriv hvilken betydning kompetanseoppbyggingen vil ha for fremtidig verdiskaping hos medvirkende bedrifter og relater dette til strategier og planer hos bedriftene.
	\item[b)] Beskriv også hvilken betydning komepetanseoppbyggingen vil ha for norsk næringsliv generelt. 
\end{itemize}

\part{Other aspects}

\section{Other socio-economic benefits}
Prosjektets betydning for norsk næringslivet og for nasjonalkompetanseoppbygging skal være beskrevet i del 3. Beskriv her hvilken betydning prosjektet kan ha utover dette. 

\section{Dissemination and communication of results}
Beskriv hvordan man planlegger å kommunisere resultater fra FoU-prosjektet.

\section{Environmental impact}
Det skal redegjøres for om prosjektgjennomføringen og/eller utnyttelse av resultatene fra FoU-prosjektet vil ha miljøkonsekvenser av betydning (positive og negative).

\section{Ethical perspectives}
Det må tas stilling til om det er etiske problemstillinger knyttet til gjennomføringen av prosjektet. Dersom det er det, skal det beskrives hvordan disse vil bli ivaretatt. De nasjonale forskningsetiske komiteenes forskningsetiske sjekkliste bør benyttes under utarbeidelse av søknaden:
\url{http://www.etikkom.no/no/Forskningsetikk/Etiske-retningslinjer/Forskningsetisk-sjekkliste/}.

\section{Gender issues}
Dersom prosjektet vil bidra til Forskningsrådets generelle mål om rekruttering av kvinner eller kjønnsbalanse i prosjekter, skal dette beskrives her. Dersom kjønnsperspektivet er relevant for innholdet i forskningen, beskriv hvordan dette hensynet er ivaretatt.

\section{Additional information specifically requested in the call}
Her skal det kun gis opplysninger som utlysningen eksplisitt har henvist til dette punktet.

\end{document}
